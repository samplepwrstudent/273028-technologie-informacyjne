%Wojciech Suchecki ITE

\documentclass{article}
\usepackage{polski}
\usepackage[utf8]{inputenc}
\usepackage{graphicx}
\usepackage{hyperref}
\usepackage{longtable}
\usepackage[13]{csvsimple}
\usepackage{pgfplots}

\title{Latech}
\author{Wojciech Suchecki}
\date{Grudzień 2022}

\begin{document}
\maketitle

\section{Wprowadzenie teoretyczne}
Spadek swobodny – ruch odbywający się wyłącznie pod wpływem ciężaru (siły grawitacji), bez oporów ośrodka. Określone wzorem:

\begin{equation}
    v = g * t
\end{equation}

\section{Opis eksperymentu}
\begin{figure}[ht]
\begin{center}
    \includegraphics{spadek_swobodny.png}
    \caption{Graficzne przedstawienie spadku swobodnego}
    \label{Spadek}
\end{center}
\end{figure}

Jak widać na rysunku pierwszym, (Rysunek \ref{Spadek}) spadek swobodny jest zjawiskiem ciekawym.

\section{Wyniki pomiarów}
Wyniki można przedstawić w postaci tabeli:
\begin{longtable}{c|c|c|c}
  \caption{Dane przedstawiajace przebieg eksperymentu}
  \label{tab:csv_data} \\
  \bfseries t[s] & \bfseries s[m] & \bfseries delta s & \bfseries \^s  \\
  \hline
  \csvreader[/csv/separator=semicolon]{dane_lab.csv}{1=\a,2=\b,3=\c,4=\d}{\a & \b & \c & \d\\}
\end{longtable}

\begin{tikzpicture}

\begin{axis}[
    title=Wykres przedstawiajacy dane z tabeli,
    xlabel= t\{s\},
    ylabel= s\{m\},
    legend pos=north west,
]


\addplot table [/pgf/number format/read comma as period, header=false,x index=0,y index=3, mark=*, col sep=semicolon] {dane_lab.csv};
\addlegendentry{Pomiary spadku}

\addplot [line width=2mm, color=green] table [/pgf/number format/read comma as period, header=false,x index=0,y index=1, mark=none, col sep=semicolon] {dane_lab.csv};
\addlegendentry{Spadek wedlug wzoru}

\end{axis}
\end{tikzpicture}

\section{Wnioski}
Na każde ciało działa grawitacja.

\end{document}


